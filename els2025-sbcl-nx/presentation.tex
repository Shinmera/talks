\documentclass[14pt,t,aspectratio=169]{beamer}
\usepackage{fontspec}
\usepackage{color}
\usepackage{minted}
\usepackage{array}
\usepackage{emoji}
\usepackage{svg}
\usepackage[absolute,overlay]{textpos}

%% These fonts are non-free.
%% Comment out the lines if you don't have them.
\setmainfont{Equity Text A}
\setsansfont{Concourse T3}
\setmonofont{Triplicate T4}

\definecolor{codecolor}{RGB}{50,50,50}
\definecolor{bgcolor}{RGB}{255,240,240}
\definecolor{red}{RGB}{224,35,35}
\setbeamercolor{background canvas}{bg=bgcolor}
\setbeamercolor{normal text}{fg=black}
\setbeamercolor{titlelike}{fg=black}
\setbeamercolor{itemize item}{fg=red}
\setbeamercolor{enumerate item}{fg=red}
\setbeamertemplate{itemize items}[circle]
\setbeamertemplate{navigation symbols}{}
\newminted[lispcode]{common-lisp}{fontsize=\footnotesize}
\newminted[smalllispcode]{common-lisp}{fontsize=\scriptsize}
\def\code#1{{\color{codecolor}\texttt{#1}}}
\renewcommand{\theFancyVerbLine}{\color{darkgray}\large \oldstylenums{\arabic{FancyVerbLine}}}

\usebackgroundtemplate{\includegraphics[width=\paperwidth,height=\paperheight]{background}}
\begin{document}
{\usebackgroundtemplate{\includegraphics[width=\paperwidth,height=\paperheight]{firstpage}}
  \begin{frame}
    \color{white}
    \vspace{3cm}
    {\hspace{1.4cm} \LARGE Porting SBCL to the Nintendo Switch} \\
    \vspace{1cm}
    {\hspace{1.6cm} Yukari Hafner, Charles Zhang} \\
    \vspace{0.2cm}
    {\hspace{2.1cm}\texttt https://shirakumo.org}
  \end{frame}}

\begin{frame}
  \frametitle{The Device}
  \includegraphics[height=5cm]{switch}
  \begin{itemize}
  \item CPU: ARM 4 Cortex-A57 64-bit
  \item OS: ``Horizon OS'', proprietary micro-kernel
  \item SDK: C++, proprietary version of Clang
  \end{itemize}
\end{frame}

\begin{frame}
  \frametitle{Immediate Challenges}
  \begin{itemize}
  \item Everything is proprietary and under NDA \\
    \Rightarrow{} Scarce public information
  \item The OS is not BSD or even fully POSIX \\
    \Rightarrow{} Need new OS abstractions
  \item There are no inter-thread signals \\
    \Rightarrow{} Can't use usual GC tricks
  \item We are not allowed to create executable pages \\
    \Rightarrow{} No compilation at runtime
  \item The OS enforces strict ASLR \\
    \Rightarrow{} Can't hard-wire object addresses
  \end{itemize}
\end{frame}

\begin{frame}
  \frametitle{Basic Ideas}
  \begin{itemize}
    \color{lightgray}
  \item Everything is proprietary and under NDA \\
    \textcolor{red}{\Rightarrow} \textcolor{black}{Only publicise our own interfaces}
  \item The OS is not BSD or even fully POSIX \\
    \textcolor{red}{\Rightarrow} \textcolor{black}{Write C(++) shim libraries for access}
  \item There are no inter-thread signals \\
    \textcolor{red}{\Rightarrow} \textcolor{black}{Use safepoints}
  \item We are not allowed to create executable pages \\
    \textcolor{red}{\Rightarrow} \textcolor{black}{Compile everything on linux
      and shrinkwrap}
  \item The OS enforces strict ASLR \\
    \textcolor{red}{\Rightarrow} \textcolor{black}{Make all code and data fully
      relocatable}
  \end{itemize}
\end{frame}

\begin{frame}
  \frametitle{History of the Project}
  \begin{itemize}
  \item Initial discussion and runtime port at ELS 2022 \\
  \item Decided to seriously pursue project at ELS 2023 \\
    \Rightarrow{} Sketched out project plan in the airport afterwards
  \item Most of the initial work and design problems solved in the
    following months \\
  \item Debugging GC issues and improving debugging \\
    experience in 2024 \\
  \item Game engine examples stabilized, Kandria running \\
    in early 2025 \\
  \end{itemize}
\end{frame}

\begin{frame}
  \frametitle{A Standard SBCL Build}
  \begin{itemize}
  \item build-config \\
    \Rightarrow{} Gather system info
  \item make-host-1 \\
    \Rightarrow{} Emit C headers and support files
  \item make-target-1 \\
    \Rightarrow{} Compile the C runtime on the target
  \item make-host-2 \\
    \Rightarrow{} Cross-compile the compiler on the host
  \item make-target-2 \\
    \Rightarrow{} Use the compiler from the host to\\
    \quad incrementally compile the rest on the target
  \end{itemize}
\end{frame}

\begin{frame}
  \includegraphics[height=8.5cm]{build.png}
\end{frame}

\begin{frame}
  \frametitle{Relocation}
  \begin{itemize}
  \item Lisp objects are full of absolute pointers to other objects
  \item Lisp objects live in Lisp spaces mapped at fixed addresses
  \item Saving an image dumps Lisp spaces to disk verbatim
  \item Reloading an image is fast; just \texttt{mmap} the on-disk \\
    spaces into the process. \\
    \textcolor{red}{\Rightarrow} Problem: enforced ASLR means no means \\
    \quad to map spaces to desired addresses
  \end{itemize}
\end{frame}

\begin{frame}
  \frametitle{Relocation (cont.)}
  \begin{itemize}
    \item Solution: Relocate all Lisp spaces on start up \\
      \textcolor{red}{\Rightarrow} Problem: Code generation hardwires some primitive \\
      \quad object addresses \\
      \Rightarrow{} Solution: Change codegen to be position independent
    \item Code objects in SBCL also contain Lisp pointers \\
      \textcolor{red}{\Rightarrow} Problem: Relocation and GC needs to fix up \\
      \quad these constants, but code objects are not writable \\
      \Rightarrow{} Solution: Segregate code instructions and \\
      \quad data in code objects into different ELF sections, \\
      \quad rewrite code instructions to reference r/w section
  \end{itemize}
\end{frame}

\begin{frame}[fragile]
  \frametitle{Relocation example}
\begin{minted}[fontsize=\small]{common-lisp}
 (defun f ()
   (cons '(42 . 1958) 'els2025))
\end{minted}
  A Lisp function referencing the constants \code{'(42 . 1958)} and \code{'ELS2025}
\end{frame}

\begin{frame}
  \begin{figure}
    \begin{columns}
      \column{0.6\linewidth}
      \vspace{-2em}
      \phantom{123}
      \includesvg[scale=0.5]{Code object before
        shrinkwrapping.drawio.svg}
      \column{0.3\linewidth}
      Representation of the compiled code object (in boldface)
      for the function \protect\texttt{\#'f} in memory, along with the
      cons and symbol objects it references.
      \column{0.1\linewidth}
    \end{columns}
  \end{figure}
\end{frame}

\begin{frame}
    \includesvg[scale=0.5]{Code objects in memory after
      shrinkwrapping.drawio.svg}

    \setlength{\TPHorizModule}{\textwidth}
    \setlength{\TPVertModule}{\textwidth}
    \begin{textblock}{0.5}(0.41,0.25)
      Representation of the same code object and the cons and
      symbol objects it references after code/data segregation and the
      rest of the shrinkwrapping process.
    \end{textblock}
\end{frame}

\begin{frame}
  \frametitle{Garbage Collection}
  \begin{itemize}
  \item Garbage collection on SBCL on non-Windows platforms uses POSIX
    signals to stop-the-world. \\
    \textcolor{red}{\Rightarrow} Problem: No POSIX signals. \\
    \Rightarrow{} Solution: Use safepoints
  \item BUT: The safepoint mechanism still uses virtual memory \\
    hardware exceptions to stop the current thread \\
  \textcolor{red}{\Rightarrow} Problem: No ability to handle OS exceptions \\
  \Rightarrow{} Solution: Change safepoints to poll state flags
  \end{itemize}
\end{frame}

\begin{frame}
  \frametitle{Misc Issues}
  \begin{itemize}
  \item SBCL uses hardware exceptions to signal some conditions \\
    \Rightarrow{} Solution: Change to full calls to \texttt{ERROR} and \texttt{SIGNAL}
  \item CLOS requires runtime compilation for dispatch JIT \\
    \Rightarrow{} Solution: Use an interpreter where absolutely necessary
  \item The existing shrinkwrap tool is not GC-safe for precise \\
    platforms like ARM64, random crashes with large cores \\
    \Rightarrow{} Solution: Complicate the build process even more \\
    \quad by using a separate SBCL with memory spaces \\
    \quad mapped away from the offline core's spaces
  \end{itemize}
\end{frame}

\begin{frame}[c]{ }
  \centering
  {\Huge Live Demo}
\end{frame}

\begin{frame}
  \frametitle{Further Work}
  \begin{itemize}
  \item Startup time is long, build times insufferably long \\
    \Rightarrow{} Figure out how to cache more stuff
  \item Optimising Trial and Kandria \\
    \Rightarrow{} Lots of profiling work that can be done on PC
  \item Porting to the Nintendo Switch 2 \\
    \Rightarrow{} As soon as plebians like us get access from almighty \\\quad Nintendo
  \end{itemize}
\end{frame}

{\usebackgroundtemplate{\includegraphics[width=\paperwidth,height=\paperheight]{firstpage}}
  \begin{frame}[c]{ }
    \centering\color{white}
    \vspace{2cm}
    {\LARGE Thank you!} \\
    \vspace{0.2cm}
    Consider supporting our work: \\
    \includegraphics[width=3cm]{patreon} \\
    {\texttt https://patreon.com/shinmera}
  \end{frame}}
\end{document}

%%% Local Variables:
%%% mode: latex
%%% TeX-master: t
%%% TeX-engine: luatex
%%% TeX-command-extra-options: "-shell-escape"
%%% End:
