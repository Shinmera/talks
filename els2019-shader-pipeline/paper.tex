\documentclass[format=sigconf]{acmart}
\usepackage[english]{babel}
\usepackage{float}
\usepackage[labelfont=bf,textfont=md]{caption}
\usepackage{graphicx}
\usepackage{xcolor}
\usepackage{minted}
\usepackage{hyperref}
\usepackage[all]{hypcap}
\usemintedstyle[glsl]{default}
\usemintedstyle[common-lisp]{default}
\newmintinline[code]{text}{}
\bibliographystyle{unsrt}

\hypersetup{
  colorlinks,
  linkcolor={red!50!black},
  citecolor={blue!50!black},
  urlcolor={blue!80!black}
}

\setcopyright{rightsretained}
\acmDOI{}
\acmISBN{}
\acmConference[ELS'19]{the 12th European Lisp Symposium}{April 1--2 2019}{%
  Genova, Italy}

\begin{document}

\title{Shader Pipeline and Effect Encapsulation using CLOS}

\author{Nicolas Hafner}
\affiliation{%
  \institution{Shirakumo.org}
  \city{Zürich}
  \country{Switzerland}
}
\email{shinmera@tymoon.eu}

\begin{abstract}
  %% FIXME: Abstract
\end{abstract}

\begin{CCSXML}
  %% FIXME: CCS
\end{CCSXML}

\keywords{Common Lisp, GLSL, OpenGL, GPU, CLOS, Object Orientation}

\maketitle

\newpage

\def\abovecaptionskip{1pt}
\def\listingautorefname{listing}
\def\figureautorefname{figure}

\section{Introduction}\label{introduction}


\section{Related Work}\label{relatedwork}
Courreges\cite{gtav} presents an in-depth analysis of the rendering procedure employed by the modern, high-production game GTA V. It illustrates the many stages to produce a final image, as well as their data dependencies. \\

Harada et al.'s work on Forward+\cite{forward+}\cite{forward+talk} also clearly illustrates the need for systems that support multi-staged rendering pipelines with complex data interaction schemes. \\

Gyrling\cite{fibers} presents an overview of the techniques used to perform parallel rendering in Naughty Dog's commercial engine. Individual steps within a stage, render stages of a frame, and multiple frame renderings are divided up into many small jobs that can run in parallel and are synchronised using counters on a shared structure. \\

The case study of the Unity game engine by Messaoudi et al\cite{unity} shows the availability of a set of fixed rendering pipelines that can be customised in a very limited extent with custom shaders. However, these shaders must fit into Unity's existing lighting and overall rendering model. While Unity does allow building a custom pipeline via their Scriptable Rendering Pipeline\cite{unitycustom}, they do not seem to offer any specific encapsulation or modularity features. \\



\section{Overview}\label{overview}

\section{Passes}\label{passes}

\section{Pipelines}\label{pipelines}

\section{Allocation}\label{allocation}

\section{Conclusion}\label{conclusion}


\section{Further Work}\label{furtherwork}


\section{Acknowledgements}\label{acknowledgements}


\section{Implementation}\label{implementation}
An implementation of the proposed system can be found at
\\\href{https://github.com/Shirakumo/trial/blob/f34a79f0a6df21d1ed9259e85fbb3c7eed39352b/shader-pass.lisp}{https://github.com/Shirakumo/trial/blob/\\f34a79f0a6df21d1ed9259e85fbb3c7eed39352b/shader-pass.lisp}
\\\href{https://github.com/Shirakumo/trial/blob/f34a79f0a6df21d1ed9259e85fbb3c7eed39352b/pipeline.lisp}{https://github.com/Shirakumo/trial/blob/\\f34a79f0a6df21d1ed9259e85fbb3c7eed39352b/pipeline.lisp}
\\\url{https://github.com/Shinmera/flow} \\

A more in-depth discussion of the system can be found at
\\\url{https://reader.tymoon.eu/article/363}
\\\url{https://reader.tymoon.eu/article/364}

\bibliography{paper}
\end{document}

%%% Local Variables:
%%% mode: latex
%%% TeX-command-extra-options: "-shell-escape"
%%% TeX-master: t
%%% TeX-engine: luatex
%%% End:
