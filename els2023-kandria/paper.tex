\documentclass[format=sigconf]{acmart}
\usepackage[utf8]{inputenc}
\usepackage{enumitem}
\usepackage{float}
\usepackage[labelfont=bf,textfont=md]{caption}
\usepackage{graphicx}
\usepackage{xcolor}
\usepackage{minted}
\usepackage{hyperref}
\usepackage[parfill]{parskip}
\usepackage[all]{hypcap}
\usemintedstyle[common-lisp]{default}
\newmintinline[code]{text}{}
\bibliographystyle{plainnat}

\hypersetup{
  colorlinks,
  linkcolor={red!50!black},
  citecolor={blue!50!black},
  urlcolor={blue!80!black}
}

\newlist{step}{enumerate}{10}
\setlist[step]{label*=\arabic*.,leftmargin=2em}

\acmConference[ELS'23]{the 15th European Lisp Symposium}{March 21--22 2023}{%
  }
\acmISBN{}
\acmDOI{}
\setcopyright{rightsretained}
\copyrightyear{2023}

\begin{document}

\title{Experience Report: Kandria - A Game in Common Lisp}

\author{Nicolas Hafner}
\affiliation{%
  \institution{Shirakumo.org}
  \city{Zürich}
  \country{Switzerland}
}
\email{shinmera@tymoon.eu}

\begin{abstract}
  
\end{abstract}

\begin{CCSXML}
  
\end{CCSXML}

\keywords{Common Lisp, Games, Video Games, Computer Graphics, Experience Report}

\maketitle

\def\abovecaptionskip{1pt}
\def\listingautorefname{Listing}
\def\figureautorefname{Figure}

\section{Introduction}\label{introduction}

\section{Related Work}\label{relatedwork}

\section{Missing Components}\label{yaks}

\section{Garbage Collection}\label{gc}

\section{Performance}\label{performance}

\section{Deployment}\label{deployment}

\section{Conclusion}\label{conclusion}

\section{Further Work}\label{further-work}
Currently a sizable amount of the work in Kandria has not been backported into Trial for more general purpose use. We would like to extract a few of the systems and generalise them to make them available for other users.

We are also working on implementing several new subsystems in Trial to allow creating 3D games, as well. A skeletal animation system has been completed, and we're currently working on a physics subsystem. Also needed will be several spatial query structures to speed up collision testing.

Finally we are also exploring the possibility of porting the engine to work on closed platforms such as the Nintendo Switch. This presents several challenges that we unfortunately cannot elaborate on here due to non-disclosure agreements.

At some point we would like to get rid of the dependency on GLFW to reduce further dependence on C libraries and the issues arising from them. However, GLFW has proven extremely stable in this regard, and as such this is of very low priority for us.

\section{Acknowledgements}\label{acknowledgements}
We would like to thank \textit{you} for being beautiful and nice :)
\bibliography{paper}
\end{document}

%%% Local Variables:
%%% mode: latex
%%% TeX-command-extra-options: "-shell-escape"
%%% TeX-master: t
%%% TeX-engine: luatex
%%% End:
