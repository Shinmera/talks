\documentclass[format=sigconf]{acmart}
\usepackage[utf8]{inputenc}
\usepackage{enumitem}
\usepackage{float}
\usepackage[labelfont=bf,textfont=md]{caption}
\usepackage{graphicx}
\usepackage{xcolor}
\usepackage{minted}
\usepackage{hyperref}
\usepackage[parfill]{parskip}
\usepackage[all]{hypcap}
\usemintedstyle[common-lisp]{default}
\newmintinline[code]{text}{}
\bibliographystyle{plainnat}

\hypersetup{
  colorlinks,
  linkcolor={red!50!black},
  citecolor={blue!50!black},
  urlcolor={blue!80!black}
}

\newlist{step}{enumerate}{10}
\setlist[step]{label*=\arabic*.,leftmargin=2em}

\acmConference[ELS'22]{the 15th European Lisp Symposium}{March 21--22 2022}{%
  Porto, Portugal}
\acmISBN{978-2-9557474-3-8}
\acmDOI{10.5281/zenodo.2636508}
\setcopyright{rightsretained}
\copyrightyear{2022}

\begin{document}

\title{Lichat, A Lightweight Chat Protocol}

\author{Nicolas Hafner}
\affiliation{%
  \institution{Shirakumo.org}
  \city{Zürich}
  \country{Switzerland}
}
\email{shinmera@tymoon.eu}

\begin{abstract}
  In this paper we present Lichat, a simple and lightweight chat protocol. We take particular interest in how relying on Common Lisp as a base language led to a fast prototyping of the protocol and rapid evolution of its design, and then evaluate the consequences of these decisions when porting the protocol to other languages such as Java, JavaScript, Python, and Elixir.
\end{abstract}

\begin{CCSXML}
  <ccs2012>
  <concept>
  <concept_id>10003033.10003039.10003051</concept_id>
  <concept_desc>Networks~Application layer protocols</concept_desc>
  <concept_significance>500</concept_significance>
  </concept>
  <concept>
  <concept_id>10002951.10003260.10003304</concept_id>
  <concept_desc>Information systems~Web services</concept_desc>
  <concept_significance>500</concept_significance>
  </concept>
  <concept>
  <concept_id>10011007.10011074.10011092.10010876</concept_id>
  <concept_desc>Software and its engineering~Software prototyping</concept_desc>
  <concept_significance>300</concept_significance>
  </concept>
  </ccs2012>
\end{CCSXML}

\ccsdesc[500]{Networks~Application layer protocols}
\ccsdesc[500]{Information systems~Web services}
\ccsdesc[300]{Software and its engineering~Software prototyping}

\keywords{Common Lisp, Chat, Protocol, Design, Networking}

\maketitle

\newpage

\def\abovecaptionskip{1pt}
\def\listingautorefname{Listing}
\def\figureautorefname{Figure}

\section{Introduction}\label{introduction}
The IRC\cite{oikarinen1993internet} (Internet Relay Chat) protocol has long stood the test of time as a reliable protocol for text chat communication. Large networks and a plethora of mature clients still exist and are in use today. However, the IRC protocol is also plagued by a number of issues that stem from its historical design and largely distributed evolution. In particular, the protocol has many different, incompatible extensions, and features many idiosyncrasies that make it difficult to write new clients or servers.

IRC has also failed to catch up with several changes in user expectations for chat protocols today. It lacks support for custom emoticons, search, formatting, does not permit multiple connections to the same user, and does not even offer an account registration or authorisation process. Instead the onus is on the servers to offer extensions, and on the clients to implement such features completely locally (forgoing synchronisation).

Over time many other chat systems such as AIM, ICQ, and Skype have come and largely gone, though all of these focus on a different model of operation than IRC. They focus on a friends-list based approach where people talk to each other directly, rather than relying on sometimes huge public chat rooms.

More recently, proprietary systems such as Slack, Discord, and Microsoft Teams have sprung up that follow the IRC model more closely, though all focus on a more sheltered experience where communities are gated off via separate ``servers'' in which the channels reside. People cannot see channels from other servers without being invited. All of these solutions use proprietary, closed-source protocols and often even prohibit the usage of third-party clients, severely restricting user-freedom and endangering the longevity of the platform.

On the open source side, the Matrix (formerly Riot) protocol has been trying to offer a modernised, federated approach to the IRC model. Being federated and open it has a much higher chance at staying around for decades to come than many of the aforementioned proprietary solutions. However, just like the proprietary counterparts, the Matrix protocol is complex and far from trivial to implement.

With Lichat we instead focused on developing a protocol that should be trivial to implement a client for, and not too hard to implement a server for. We also focused on a design that would avoid many of the pitfalls we found in IRC, and in general strives for a few basic properties that ensure reliable and easy to understand communication between the client and server. The core protocol of Lichat is very small, only covering the most basic elements, instead relying on protocol extensions to provide additional functionality that bring it up to par with competing systems such as Discord or Matrix.

In \autoref{initial-design} we present our initial prototyping for the protocol and illustrate the advantages of picking Lisp as a starting point. In \autoref{ratified-design} we discuss how the design evolved after the initial prototyping phase and discuss the porting of the protocol implementations to other languages. Finally, in \autoref{extensions} we discuss the use and development of extensions to provide additional functionality.

\section{Related Work}\label{relatedwork}
Penpoppichanan\cite{penpoppichanan2001security} identifies a number of security and privacy related issues with the IRC protocol. While Lichat shares the basic concerns about privacy, being an unencrypted protocol, it does fix several of the identity and authentication related issues identified in the thesis.

The XMPP core specification\cite{saint2004extensible} alone is over 200 pages long, and does not include the already non-trivial question of parsing XML, which is used for its wire format. Saint-Andre\cite{saint2009xmpp} also points out the lack of a proper end-to-end encryption scheme for the protocol.

Jennings et al.\cite{jennings2006study} compare a variety of Instant Messenger systems such as AIM and MSN, which are all now largely defunct. The paper focuses primarily on the distribution of sessions and the scalability of the networks. Lichat is explicitly focused on small to mid-sized communities and is thus not concerned with the scalability to tens of thousands of simultaneous users.

The Matrix core specification\cite{matrix} is extremely extensive. It covers a lot of ground including encryption, federation, and other advanced features supported by the protocol. Especially notable is that the protocol is quite complex and features a lot of out-of-stream APIs, whereas Lichat handles everything within the single stream between the user and the server.

\section{Initial Design}\label{initial-design}
As a basis for the protocol we chose to use UTF-8 character streams for the transport. This choice ensures that protocol messages can be read and written by a human, which eases development and debugging. Specifying the character encoding also avoids ambiguity when interpreting the underlying bytes. For the actual data format, we relied on Common Lisp s-expressions, which could simply be \code{read} and \code{print}ed using the native Lisp functions.

Doing so both eliminated the need to write encoders for the prototype, and ensured that we had a data structure basis that could support more complex data payloads further down the line. The actual message data types communicated with were also directly represented using Common Lisp standard classes, allowing for tight encapsulation of data attributes, and the natural combination thereof through multiple inheritance.

Each message to be sent was trivially encoded by printing a list consisting of the message's type and its construction arguments. The decoding then followed just as trivially. Doing so meant we could set up a server-client communication with as close to no effort as possible while remaining extensible for future changes. Particularly, the choice of symbols for message type names and argument names allowed us to use packages to isolate additions made by extensions into their own namespaces, avoiding future conflicts.

Another of the issues with IRC's design is that it is not strictly possible to verify whether messages arrived at the server at all, and if a response is given, which request that response belongs to. To rectify this issue, we require the inclusion of an \code{:id} field in every message sent, which the server will either re-use on a mirror-reply, or include as a reference in case of a failure. We also require the server to always reply to a message with some form of a response, even if it is just sending the original message back to the client.

The client should also include a \code{:clock} field of the local Universal-Time at the time of sending. This field ensures that, if distributed to other users, they can get an accurate ordering of events that reflects the sending user's point of view, without potential network delays changing the sequence of events. It also allows the server and clients to estimate lag in the network and issue warnings on unstable connections, or terminate them early, ensuring an ultimately more stable communication.

We also require the inclusion of a \code{:version} field on connection to allow the server to reject the connection or switch to a compatibility mode. We also include a \code{:extensions} field to allow the listing of supported protocol extensions from the client, which the server then modifies with its own set of supported extensions when confirming the connection. This way both server and client are fully aware of supported extensions that they should deal with.

On the client to client communication side, we decided to enforce all communication between clients to occur over channels. This decision simplifies the design on both sides, as there don't need to be any special rules to deal with direct communication. With all of this groundwork laid, we could implement a basic chat server that allowed for communication over multiple channels, as shown in \autoref{lst:basic-exchange}.

\begin{listing}[h]
\begin{minted}[fontsize=\scriptsize]{common-lisp}
> (connect :id 0 :clock 5243 :from "tester" :version "1.0" :extensions ())
< (connect :id 0 :clock 5243 :from "tester" :version "1.0" :extensions ())
> (join :id 1 :clock 5243 :from "tester" :channel "lichatters")
< (join :id 1 :clock 5243 :from "tester" :channel "lichatters")
< (message :id 37 :clock 5244 :from "shinmera" :channel "lichatters" :text "Welcome!")
> (message :id 2 :clock 5246 :from "tester" :channel "lichatters" :text "Hiya")
< (message :id 2 :clock 5246 :from "tester" :channel "lichatters" :text "Hiya")
\end{minted}
\caption{A basic protocol exchange.}
\label{lst:basic-exchange}
\end{listing}

After these changes most of the protocol design revolved around the implementation of a permissions system to allow channel operators better control over what users are allowed to do, and the ratification of distinct error types for potential failures.

The permissions system in general is very simple: each channel holds a map where each update type is associated with either a list of permitted users, or a list of denied users. As all communication between users must occur through channels, this system provides a simple, but expressive model to handle moderation. For instance, banning a user simply requires removing their ability to send a \code{join} message to the channel.

For failures, a simple hierarchy was designed:

\begin{step}
\item \code{failure} Any kind of issue
  \begin{step}
  \item \code{malformed-update} The message could not be read at all
  \item \code{update-too-long} The message was too long and was skipped wholesale
  \item \code{update-failure} An issue with a request the user sent
    \begin{step}
    \item \code{insufficient-permissions} The message was denied
    \item \code{no-such-channel} The requested channel does not exist
    \item ...
    \end{step}
  \end{step}
\end{step}

Particularly, \code{update-failure}s carry a field called \code{:update-id} which refers back to the ID of the original request, allowing the client to correlate the failure with the previous request, avoiding the issue of confusing responses should multiple requests and responses be in flight at the same time.

Finally, in order to ensure that permissions tables actually have any grip at all, and to allow users to use multiple connections at once, we introduced an accounts system against which the user is authenticated when connecting.

The initial client and server implementations were fully in Common Lisp, with a JavaScript client intended for public use. However, we ran into scaling and concurrency issues with the Common Lisp server that proved extremely difficult to debug and understand, as they often took days on the public server to be triggered. The lack of primitives for efficient asynchronous programming and lightweight processes pushed us to attempt a rewrite of the server in Elixir instead, once the core protocol was ratified fully.
      
\section{Ratified Design}\label{ratified-design}
After the initial proof of concept, we decided to simplify the protocol in order to ease the porting process. The first part was to reduce the s-expression syntax to make the writing of a compliant parser easier, and safer. To this end, we restricted the possible data types to the following:

\begin{step}
\item \code{number}
  \begin{step}
  \item \code{integer}
  \item \code{float}
  \end{step}
\item \code{symbol} (names are case-insensitive, no gensyms)
  \begin{step}
  \item \code{keyword}
  \item \code{boolean}
  \end{step}
\item \code{list}
\item \code{string}
\end{step}

Further, we required the use of a null byte as an end of message marker. The null byte is useful so that a server or client can restore the stream of messages should a parser failure occur, or should the update be rejected due to excessive length. We also specified that an optional, missing field is equivalent to specifying the field to be \code{nil}, and vice-versa. This aliasing simplifies initialising and dealing with optional values a lot.

To ease the initiation and maintenance of the connection, the \code{:from} and \code{:clock} fields were made optional, instead requiring the server to automatically fill them in should they be missing. This behaviour makes it even more trivial to write a simple, first client.

We decided to specify two distinctions in channel use from regular channels: the ``primary'' channel, and ``anonymous'' channels. The primary channel is used to keep track of all users on the server, as all users are automatically joined to this channel after connecting. Anonymous channels are used for private ``direct'' communications between users. They are named such as the name is picked automatically by the server, and they won't appear in channel listings. Otherwise they operate exactly the same as any other channel, making it trivial for clients to implement private user to user communication, which can also be extended to private group chats by inviting other users.

The primary channel is particularly useful to manage permissions, as all messages that do not target a specific channel instead default to having the access checked against the primary channel.

Finally, we ratified the definition of all message types in a machine-readable format (\autoref{lst:spec}) that can be parsed using the same parser of the wire format. These machine-readable specs allow protocol implementations to easily parse up-to-date representations of the entire message type hierarchy.

\begin{listing}
\begin{minted}[fontsize=\footnotesize]{common-lisp}
(define-package "lichat")

(define-object lichat:update ()
  (:id id)
  (:clock integer :optional)
  (:from string :optional))

(define-object lichat:channel-update (lichat:update)
  (:channel string))

(define-object lichat:text-update (lichat:update)
  (:text string))

(define-object lichat:message (lichat:channel-update lichat:text-update))
\end{minted}
\caption{A section of the machine-readable protocol specification}
\label{lst:spec}
\end{listing}

The core protocol specification has been incredibly stable since these changes, only requiring minor amendments to clarify and disambiguate certain points. Almost all of the active development since happened within extensions and their specifications, instead.

\section{Porting}
We implemented client software in JavaScript, Java, and Python, as well as a server in Elixir. For each of those the porting process for the basic syntax of the protocol was fairly simple, only requiring a recursive descent parser that is trivial enough to write by hand. Most of the issues arose when porting the message class hierarchy.

Aside from the wire protocol and message type hierarchy, each client follows a pretty simple scheme as outlined in \autoref{lst:client}. Particularly simplifying is the fact that protocol-level pings are sent by the server, removing the need for complex timeout tracking in the client.

\begin{listing}
\begin{minted}[fontsize=\footnotesize]{common-lisp}
(defun connect ()
  (socket-connect)
  (send 'connect :from username)
  (let ((reply (make-update (read-list))))
    (etypecase reply
      (connect
       (setf username (from reply))
       (setf supported-extensions (extensions reply))))))

(defun skip ()
  (loop until (= 0 (read-byte)))
  (continue))

(defun reconnect ()
  (socket-disconnect)
  (connect)
  (continue))

(defun message-loop ()
  (loop (with-simple-restart (continue)
          (handle (make-update (read-list))))))

(defun main ()
  (connect)
  (handler-bind ((syntax-error #'skip)
                 (update-malformed #'continue)
                 (timeout #'reconnect))
    (message-loop)))
\end{minted}
\caption{A pseudo-code example of a client's main loop}
\label{lst:client}
\end{listing}

In Python we were able to make use of multiple superclasses as we would in Common Lisp, keeping the implementation quite straight-forward.

JavaScript does not support multiple inheritance out of the box, so we implemented our own scheme following the serialised inheritance model of CLOS. This scheme does however still not integrate neatly with the native JavaScript type system, meaning we also require the use of a specialised type test for dispatch.

Similarly to JavaScript, Java does not support multiple inheritance. Instead of defining an ad-hoc system however, we opt for a stubbing approach where we create a single inheritance chain, instead transforming extraneous types into additional class fields. The Java implementation is further restricted by its static typing requirements, forcing additional logic when parsing concrete class instances from the message representation. Finally, Java does not support runtime creation of classes, requiring us to write a code generator that emits static Java code, instead of creating the classes at runtime like the Python and JavaScript implementations.

Our Elixir server implementation eschews inheritance altogether, as the language does not share an object-oriented system at all. We instead manually flatten the inheritance by specifying all inherited fields on each message type. Each message type is represented through a module, with a function to handle the update on the server-side. Thanks to Elixir's macro system we were able to avoid most of the boilerplate overhead of writing each message type, but we have so far not implemented automated creation of the message representations from the machine-readable specification.

\section{Extensions}\label{extensions}
While the core protocol only provides text-only chat, we have specified and implemented a number of extensions that provide additional functionality commonly found in more modern systems such as Discord.

We've specified over twenty extensions at the time of writing. Instead of outlining them all, we will instead name a few of particular import.

A particular lament with IRC is the lack of easy file sharing. Users have come to expect being able to share images, videos, and other small files directly through the chat platform. IRC does have the DCC (Direct Client to Client) protocol extension, but it only allows exchanging files between two users directly, rather than broadcasting it to a channel. The ``shirakumo-data'' extension introduces a new message type that allows sending a base64-encoded payload, with an associated mime-type. The server can restrict permitted mime-types, and the general message length checking (with the null-byte skipping mechanism) allows dropping data payloads that are too large. The ``shirakumo-link'' extension further optimises sending media by letting the server save data-payloads and distribute them via an internet link instead, reducing the amount of data needing to be sent out to the channel drastically.

The ``shirakumo-channel-info'' and ``shirakumo-user-info'' extensions in turn allow associating metadata with channels and user accounts. Like with the data extension, the server controls what metadata can be stored and how much of it. These extensions allow storing common things such as channel rules, a topic, icons, or other user contact data.

With the ``shirakumo-emotes'' and ``shirakumo-reactions'' extensions, users can upload custom emoticons for their channels and use them in messages, or as ``reactions'' to other user's messages. These features have become especially popular on platforms such as Discord and Slack.

Finally, the ``shirakumo-history'' extension allows the server (if requested) to store the chat logs of a channel, and then allows users to search through the history of a channel, and also in combination with the ``shirakumo-backfill'' extension request a replay of the most recent messages after joining a channel, to get some context for the conversation.

\section{Conclusion}\label{conclusion}
We have demonstrated a protocol design that offers a solid base for a chat system, which is both simple to implement, and easily extensible to provide advanced features that elevate it beyond simple text chat.

With our extension in place, user experience is very close to that offered by Discord, Slack, or Matrix, with the primary difference being that, as the protocol currently stands, a user account is always local to a specific server, rather than being a part of the global network of servers.

Our base protocol design has proven very robust, even in the face of these many extensions and the addition of multiple new clients across different languages.

\section{Further Work}\label{further-work}
On the protocol side, we would like to develop further extensions, primarily to allow for VOIP, federation, and encrypted messaging. Encryption and privacy are of especially big interest, though are quite challenging, as end-to-end encrypted one-to-many communication is difficult to achieve efficiently\cite{nabeel2017many}\cite{whisper2014}.

We would also like to develop further client implementations, particularly for C/C++ for use in the multi-protocol client Pidgin.

\section{Acknowledgements}\label{acknowledgements}
We would like to thank Georgiy Tugai, Jan Moringe, Bart Botta, and Selwyn Simsek for their feedback on this paper and contributions to Lichat's development.

\section{Source Code}\label{source}
The lichat specification and implementations can be found freely on the web:
\begin{itemize}
\item Lichat \url{https://shirakumo.org/projects/lichat}
\item JavaScript \url{https://shirakumo.org/projects/lichat-js}
\item Elixir \url{https://shirakumo.org/projects/ex-lichat}
\item Java \url{https://shirakumo.org/projects/jlichat}
\item Python \url{https://shirakumo.org/projects/py-lichat}
\end{itemize}

\bibliography{paper}
\end{document}

%%% Local Variables:
%%% mode: latex
%%% TeX-command-extra-options: "-shell-escape"
%%% TeX-master: t
%%% TeX-engine: luatex
%%% End:
